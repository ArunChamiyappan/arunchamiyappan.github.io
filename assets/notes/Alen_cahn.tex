\documentclass[12pt,a4paper]{article}
\usepackage{amsmath, amssymb}
\usepackage{geometry}
\geometry{margin=1in}

\title{The Allen--Cahn Equation}
\author{}
\date{}

\begin{document}

\maketitle

\section*{Allen--Cahn Equation in Materials Science}

The \textbf{Allen--Cahn equation} is a phase-field model that describes the temporal evolution of an 
\emph{order parameter} \(\phi(\mathbf{x},t)\), which distinguishes between different phases of a material. 
It was originally proposed to model the motion of anti-phase boundaries in alloys.

\subsection*{Free Energy Functional}

The model is based on a total free energy functional:
\begin{equation}
F[\phi] = \int_\Omega \left( \frac{\epsilon^2}{2} |\nabla \phi|^2 + W(\phi) \right) \, d\mathbf{x},
\end{equation}
where
\begin{itemize}
    \item \(\epsilon\) controls the interface width (gradient energy coefficient),
    \item \(W(\phi)\) is a double-well potential. Here we choose
    \begin{equation}
    W(\phi) = \phi (1 - \phi)(1 - 2\phi),
    \end{equation}
    which favors two stable states: \(\phi = 0\) and \(\phi = 1\).
\end{itemize}

\subsection*{Evolution Equation}

The Allen--Cahn equation is derived using time-dependent Ginzburg--Landau dynamics:
\begin{equation}
\frac{\partial \phi}{\partial t} = -M \frac{\delta F}{\delta \phi},
\end{equation}
where \(M\) is the phase-field mobility.

Evaluating the variational derivative gives:
\begin{equation}
\frac{\partial \phi}{\partial t} =
M \left( \epsilon^2 \nabla^2 \phi - \frac{dW}{d\phi} \right).
\end{equation}

For the chosen potential,
\begin{align}
W(\phi) &= \phi - 3\phi^2 + 2\phi^3, \\[6pt]
\frac{dW}{d\phi} &= 1 - 6\phi + 6\phi^2.
\end{align}

Thus, the Allen--Cahn equation becomes
\begin{equation}
\frac{\partial \phi}{\partial t} =
M \left( \epsilon^2 \nabla^2 \phi - \left( 1 - 6\phi + 6\phi^2 \right) \right).
\end{equation}

\subsection*{Physical Interpretation}

\begin{itemize}
    \item The diffusion-like term \(\epsilon^2 \nabla^2 \phi\) smooths the interface (surface tension effect).
    \item The reaction term \(-\big(1 - 6\phi + 6\phi^2\big)\) drives the system toward the bulk phases (\(\phi = 0, 1\)).
    \item The equation models \emph{interface motion driven by curvature}.
\end{itemize}

\subsection*{Applications in Materials Science}

The Allen--Cahn equation is widely applied to model microstructural evolution in:
\begin{itemize}
    \item Grain growth,
    \item Domain coarsening,
    \item Solidification processes,
    \item Phase transformation kinetics.
\end{itemize}

\end{document}
